% Kile - graf. editor
\documentclass[11pt,a4paper]{article}
\input{il2code.tex}
\usepackage{textcomp}
\usepackage[utf8]{inputenc}
\usepackage[czech]{babel}
\usepackage{fancyhdr}
\usepackage[pdftex]{graphicx}
\newcommand{\HRule}{\rule{\linewidth}{0.5mm}}
\pagestyle{fancy}
\lhead{}
\chead{}
\rhead{Barbora Skřivánková (xskriv01) a Jan Kročil(xkroci02)}
\lfoot{Dokumentace projektu do předmětu IMS}
\cfoot{}
\rfoot{\thepage\ / \pageref{LastPage}}


\begin{document}

\begin{titlepage}
\begin{center}

% Upper part of the page. The '~' is needed because \\
% only works if a paragraph has started.
\textsc{\LARGE Vysoké učení technické v Brně, FIT}\\[3cm]

% Title
\HRule \\[0.4cm]
{ \huge \bfseries Dokumentace k projektu do předmětu IMS \\[0.4cm] }

\HRule \\[3cm]

% Author and supervisor
\begin{minipage}{1.0\textwidth}
\begin{flushleft} \large
Barbora Skřivánková (xskriv01) \\
Jan Kročil (xkroci02)
\end{flushleft}
\end{minipage}


\vfill

% Bottom of the page
{\large \today}

\end{center}
\end{titlepage}

\clearpage

\tableofcontents
\clearpage

\section{Úvod}
	\subsection{Co jsme řešili a proč jsme to řešili}
	V této práci je řešena implementace tří různých generátorů pro poukázání na možnost
	generovat reálné jevy pomocí počítače.
	\subsection{Kde jsme vzali data}
	Měřili jsme v terénu velmi sofistikovanými metodami.
	\subsection{Jak jsme zjistili, jestli je model validní}
	Porovnáváním histogramů reálného jevu a výstupu našeho generátoru.
\section{Fakta}
Hypotézy, předpoklady a všechna konkrétní čísla vyskytující se v modelu.
\section{Koncepce}
--> Abstraktní model s vyznačením relevantních faktů.
Předpokládáme výhradně Královopolský tunel za bílého dne, protože bagr neplave.
Okolnosti měření, popis způsobu měření a způsobu naměřených dat.
\section{Způsob řešení}
https://www.causeweb.org/repository/statjava/Distributions.html - applety pro vyzkouseni jednotlivych rozlozeni
\section{Testování}
	\subsection{Postup experimentování a okolnosti}
	Jak jsme zjišťovali aproximace histogramů..
	\subsection{Dokumentace jednoltivých experimentů}
	\subsection{Závěr experimentů}
	Generator vjezdu do tunelu v case simulace 2 hodiny vygeneroval pocet aut odpovidajici realnemu poctu aut, projizdejicich behem mereni. (U pripadu, ktery je modelovan v prilozenem grafu jde o konkretne 1537 modelovych aut behem dvou hodin simulace oproti 1517 realnych aut behem dvou hodin mereni.)
	Co ve výsledcích má čtenář vidět...
\section{Závěr}
Jednoznačná odpověď na prvotní otázku studie.


\end{document}
